\documentclass[a4paper,UKenglish]{darts}
\usepackage{microtype}
\bibliographystyle{plainurl}
% Commands for artifact descriptions
% Written by Camil Demetrescu and Erik Ernst
% April 8, 2014

% ARTIFACT: This entire file should be used as-is; it defines standard
% headings to be included in the artifact description, and it will be 'input'
% into the document file such that you can use the environments defined
% below

\newenvironment{scope}{\section{Scope}}{}
\newenvironment{content}{\section{Content}}{}
\newenvironment{getting}{\section{Getting the artifact} The artifact 
endorsed by the Artifact Evaluation Committee is available free of 
charge on the Dagstuhl Research Online Publication Server (DROPS).}{}
\newenvironment{platforms}{\section{Tested platforms}}{}
\newcommand{\license}[1]{{\section{License}#1}}
\newcommand{\mdsum}[1]{{\section{MD5 sum of the artifact}#1}}
\newcommand{\artifactsize}[1]{{\section{Size of the artifact}#1}}



\title{Data exploration through dot-driven development (Artifact)\footnote{This artifact is a companion of the paper:  Tomas Petricek, ``Data exploration through dot-driven development'', Proceedings of the 31st European Conference on Object-Oriented Programming (ECOOP 2017), June 18-23, 2017, Barcelona, Spain.  This work was supported in part by The Alan Turing Institute under the EPSRC grant EP/N510129/1 and by the Google Digital News Initiative.}}
\titlerunning{Data exploration through dot-driven development (Artifact)} 

\author[1]{Tomas Petricek}
\affil[1]{The Alan Turing Institute, London, UK\\
 and Microsoft Research, Cambridge, UK\\
  \texttt{tomas@tomasp.net}}
\authorrunning{T. Petricek}
\Copyright{Tomas Petricek}

\subjclass{D.3.2 Very high-level languages}
\keywords{Data science, type providers, pivot tables, aggregation}

%Editor-only macros:: begin (do not touch as author)%%%%%%%%%%%%%%%%%%%%%%%%%%%%%%%%%%
\Volume{3}
\Issue{2}
\Article{1}
\RelatedConference{European Conference on Object-Oriented Programming (ECOOP 2017), June 18-23, 2017, Barcelona, Spain}
% Editor-only macros::end %%%%%%%%%%%%%%%%%%%%%%%%%%%%%%%%%%%%%%%%%%%%%%%

\begin{document}

\maketitle

\begin{abstract}
This artifact presents The Gamma, a simple programming environment for data exploration
that uses member access as the primary mechanism for constructing queries. The artifact 
consists of two parts. The user facing web-based component allows users to explore a simple
dataset of Olympic medal winners while a back-end service provides the data and evaluates
queries executed by the user.

The purpose of the artifact is to illustrate the pivot type provider, which provides a simple
way for constructing queries in a object-based programming language equipped with member 
access. The pivot type provider can be use to
construct new queries from code or using the user interface, but it also encourages the user
to modify existing code.  
\end{abstract}

\begin{scope}
The artifact is designed to provide a repeatble way of evaluating the user experience that
is provided by the pivot type provider. As discussed in the paper, the purpose of the project
is to simplify data exploration (by using a simple programming language), make it open and 
transparent (by revealing the source code) and encourage further data exploration (by making
it easy to change the source code). The presented artifact supports this claim.
  
The artifact web page contains a number of examples that illustrate the pivot type provider. When the
page loads, the examples are executed and the results appear.
For each example, the web page shows the full source code using the pivot type provider together
with several additional libraries not presented in the paper (such as \texttt{chart} and \texttt{table}),
which are used for creating visualizations.

Clicking on the ``Edit source code'' button opens an editor where the code can be modified. The
editor supports auto-completion, which is the key feature made possible by the type system of the
language and the pivot type provider. The following three sections briefly describe three scenarios
that illustrate interesting aspects of the artifact.
\end{scope}

\begin{content}
The artifact package includes:
\begin{itemize}
\item a Docker image ({\tt thegamma-ecoop17.tar}), which contains the necessary client-side
  and server-side components for running the artifact
\item an F\# source code ({\tt thegamma-source.zip}) which contains the full source code for 
  the client-side and server-side components behind the artifact
\end{itemize}
To make evaluating of the user interface easier, we provide a Docker container that
can be used to host the system. The Docker container is based on the standard {\tt fsharp} 
image supplied by the F\# Software Foundation. 

The included source code provides complete source code
for the server-side components (running as F\# code on the server) and client-side components
(translated to JavaScript using the Fable compiler). When compiling code from scratch, several
packages are downloaded prior to the compilation from standard .NET and JavaScript package
repositories NuGet and npm.
\end{content} 

\begin{getting}

The presented artifact consists of a simple web server packaged as a Docker container. When
executed the artifact can be viewed by navigating to \url{http://localhost:8889} as described
in the artifact documentation. The artifact can be also obtained and started from
Docker hub using:

\begin{verbatim}
    docker run -p 8889:80 tomasp/thegamma-ecoop17
\end{verbatim}

\noindent
The tool implemented as part of the submitted paper is also available as a JavaScript library
that is made available through the standard JavaScript package manager npm. The full
project soruce code is also hosted on GitHub. The following links can be used to obtain the latest
version:
\begin{itemize}
  \item \url{https://github.com/the-gamma/thegamma-script}
  \item \url{https://npmjs.com/package/thegamma-script}
\end{itemize}

\end{getting} 

\begin{platforms}
  The artifact is known to work on any platform running Docker version 17 using the Google Chrome
  browser. It has been 
  specifically tested using Docker version 17.03.1-ce-win12 on the Windows operating system.
  5~GB of free space on disk and at least 2~GB of free space in RAM are sufficient for running
  the artifact.
\end{platforms}

\license{MIT ({\url https://opensource.org/licenses/MIT})}

% ARTIFACT: section specifying the md5 sum of the artifact master file
% uploaded to the Dagstuhl Research Online Publication Server, enabling 
% downloaders to check that the file is the expected version and suffered 
% no damage during download.
\mdsum{1f68db3bcbd54433a62c9b54a35a5666}

% ARTIFACT: section specifying the size of the artifact master file uploaded
% to the Dagstuhl Research Online Publication Server
\artifactsize{336 MB}


% ARTIFACT: optional appendix
%\appendix

%\section{My Appendix}

% Add here any further material you would like to include. For instance, if the artifact is itself a PDF document, add it here.


% ARTIFACT: include here any additional references, if needed...

%% Either use bibtex (recommended), but commented out in this sample

%\bibliography{dummybib}

%% .. or use the thebibliography environment explicitely


\end{document}
