\documentclass{sigchi}

\CopyrightYear{2019}
\setcopyright{rightsretained}
\doi{http://dx.doi.org/10.475/123_4}
\isbn{123-4567-24-567/08/06}
\conferenceinfo{NA}{TBA}
\acmPrice{\$0.00}

\usepackage{xcolor}
\usepackage{balance}
\usepackage{graphics}
\usepackage[T1]{fontenc}
\usepackage{txfonts}
\usepackage{mathptmx}
\usepackage[pdflang={en-US},pdftex]{hyperref}
\usepackage{color}
\usepackage{booktabs}
\usepackage{textcomp}

\usepackage{microtype}
\usepackage{ccicons}

\def\plaintitle{The Gamma}
\def\plainauthor{First Author, Second Author, Third Author,
  Fourth Author, Fifth Author, Sixth Author}
\def\emptyauthor{}
\def\plainkeywords{Authors' choice; of terms; separated; by
  semicolons; include commas, within terms only; required.}
\def\plaingeneralterms{Documentation, Standardization}

\makeatletter
\def\url@leostyle{%
  \@ifundefined{selectfont}{
    \def\UrlFont{\sf}
  }{
    \def\UrlFont{\small\bf\ttfamily}
  }}
\makeatother
\urlstyle{leo}

\def\pprw{8.5in}
\def\pprh{11in}
\special{papersize=\pprw,\pprh}
\setlength{\paperwidth}{\pprw}
\setlength{\paperheight}{\pprh}
\setlength{\pdfpagewidth}{\pprw}
\setlength{\pdfpageheight}{\pprh}

\definecolor{todoColor}{rgb}{0.6,0.0,0.2}
\definecolor{linkColor}{RGB}{6,125,233}

\newcommand{\todo}[1]{\textcolor{todoColor}{[\textbf{todo}: #1]}}

\hypersetup{%
  pdftitle={\plaintitle},
% Use \plainauthor for final version.
%  pdfauthor={\plainauthor},
  pdfauthor={\emptyauthor},
  pdfkeywords={\plainkeywords},
  pdfdisplaydoctitle=true, % For Accessibility
  bookmarksnumbered,
  pdfstartview={FitH},
  colorlinks,
  citecolor=black,
  filecolor=black,
  linkcolor=black,
  urlcolor=linkColor,
  breaklinks=true,
  hypertexnames=false
}
\begin{document}

\title{\plaintitle}

\numberofauthors{1}
\author{%
  \alignauthor{Anonymous Authors\\
    \affaddr{Unseen University}\\
    \affaddr{Discworld}\\
    \email{e-mail address}}\\
}

\maketitle

\begin{abstract}
TBA
\end{abstract}

% \category{H.5.m.}{Information Interfaces and Presentation
%   (e.g. HCI)}{Miscellaneous} \category{See
%   \url{http://acm.org/about/class/1998/} for the full list of ACM
%   classifiers. This section is required.}{}{}
%
% \keywords{\plainkeywords}

\section{Introduction}

some background

\subsection{Example}

put some screenshot here?

\subsection{Our Approach and Contributions}
We present blah blah.
Our contributions are:

\begin{itemize}
\item Novel language design based on design principles
\item Open source implementation on the web with cool use cases
\item Critical evaluation
\end{itemize}

Long-term goal of making journalism better

\section{Related work}

\textbf{Visual tools.}

\textbf{Programming tools.}
Notebooks

\textbf{Journalism.}
Idyll

\section{Overview}

some walkthrough illustrating thegamma with code and screenshots

\newpage

\section{Design principles}
The design of TheGamma is based on two categories of design goals that we identify in this
section. The first group of goals follow the observation that majority of users will not be
expert programmers and can only invest limited amount of time into learning a new tool. The
second group of goals are derived from our focus on journalistic applications and the specific
challenges of this domain.

\subsection{Principles for designing simple language}
Our approach originates in a careful consideration of the challenges faced by users learning
how to use main-stream programming languages for data science scripting. \todo{There must be
some research on learning programming we can cite}

\todo{To me, some of the things below sound kind of trivial. Is that just me, or do we
need more fundamental principles?}

\textbf{A1. Allow learning from examples.} Users of tools such as spreadsheets often learn by
looking at existing problem solutions \todo{Advait's PPIG}. Our design should allow this by
making it possible to inspect and retrace steps used while solving a problem in an existing
application.

\textbf{A2. Allow learning by experimentation.} Another principle of spreadsheets that we want
to keep is the ability to experiment and see results immediately. Our design should allow users
to try invoking an operation or modifying a parameter and quickly see if this leads to the
desired results.

\textbf{A3. Choice over construction.} To minimize the amount of information that users have to
learn and remember, our system should work in a way that allows constructing programs by
choosing from options that can reasonably appear in a current context, rather than requiring
users to recall particular syntax or exact identifier name.
\todo{recognition over recall?}

\textbf{A4. Power for power users.} Some users of the system may, over time, become advanced users
and the system should support those. In other words, the upper bound on what can be achieved
should be well above the most common use cases. \todo{I think I got this idea of "boundaries" on what
is possible from some paper, but cannot recall which...}

\textbf{A5. Minimal viable scenario.} At the same time, the complex features that power users might
need should not affect the most elementary uses of the system and should remain completely hidden
until needed. In other words, the lower bound on what one needs to know to use the system for basic
tasks should be as low as possible.

\todo{I think the above would probably be better if it was just 3 principles. A1+A2 = ``learnability'',
A3 as is, A4+A5 = ``boundaries''. The length of UIST papers also tends to be quite short, so that
might be better for space reasons too.}

\subsection{Principles addressing challenges in journalism}
In addition to consideration based on our focus on non-expert users, we also identify a number
of principles that are specific to our target domain. Following a careful consideration of
current debates in journalism, we identify the following principles. \todo{we should find some
references on trends in journalism to motivate these}

\textbf{B1. Trust through transparency.} The system should allow fact checking of the analyses to
build trust. This means that viewers should be able to determine what is the source of analysed
data and how has the data been transformed.

\textbf{B2. Opening the process to readers.} Journalists are increasingly opening the way they work
to readers in order to build trust. The ability to view how an analysis has been constructed should
not be limited just to experts (say, by running a Jupyter Notebook from GitHub), but should be
available to all interested readers.

\textbf{B3. Providing meaningful engagement mechanism.} Finally, the system should provide a
mechanism through which readers can engage in a meaningful discussion. For example, it should allow
them modify parameters of a data visualization in order to show how, e.g. different choice of
countries affects the result. 

\section{Systematic description}

\subsection{Language}

Code should be visible to allow (A1)

auto-complete friendly language allows (A2)

language design addresses (A3, A4) by making easy things easy and hard
things possible (if you know the right syntax)

\subsection{Type providers}

\section{Design principles evaluation}

\section{System evaluation}

\subsection{Something measurable}

\subsection{Case studies}

\subsection{Scalability}

\section{Discussion}

\subsection{Study limitations}
exploratory in nature so we do not make any quantitative claims about effects

not comparing against other systems

\subsection{Design principles}

\subsection{Design issues}
modifying code

\newpage
TODO: Evaluating systems paper

\balance{}
\bibliographystyle{SIGCHI-Reference-Format}
\bibliography{sample}
\end{document}
